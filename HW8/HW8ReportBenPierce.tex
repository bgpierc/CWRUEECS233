% !TEX TS-program = pdflatex
% !TEX encoding = UTF-8 Unicode

% This is a simple template for a LaTeX document using the "article" class.
% See "book", "report", "letter" for other types of document.
\documentclass[12pt]{article} % use larger type; default would be 10pt

%\documentclass{amsart}
\usepackage[utf8]{inputenc} % set input encoding (not needed with XeLaTeX)
\usepackage{hyperref}
\usepackage{listings}
\usepackage{movie15}
\usepackage{graphicx}
\usepackage{tabu}

\hypersetup{
    colorlinks=true,
    linkcolor=blue,
    filecolor=magenta,      
    urlcolor=blue,
}

%%% Examples of Article customizations
% These packages are optional, depending whether you want the features they provide.
% See the LaTeX Companion or other references for full information.

%%% PAGE DIMENSIONS
\usepackage{geometry} % to change the page dimensions
\geometry{a4paper} % or letterpaper (US) or a5paper or....
% \geometry{margin=2in} % for example, change the margins to 2 inches all round
% \geometry{landscape} % set up the page for landscape
%   read geometry.pdf for detailed page layout information

\usepackage{graphicx} % support the \includegraphics command and options

% \usepackage[parfill]{parskip} % Activate to begin paragraphs with an empty line rather than an indent

%%% PACKAGES
\usepackage{booktabs} % for much better looking tables
\usepackage{array} % for better arrays (eg matrices) in maths
\usepackage{paralist} % very flexible & customisable lists (eg. enumerate/itemize, etc.)
\usepackage{verbatim} % adds environment for commenting out blocks of text & for better verbatim
\usepackage{subfig} % make it possible to include more than one captioned figure/table in a single float
% These packages are all incorporated in the memoir class to one degree or another...

%%% HEADERS & FOOTERS
\usepackage{fancyhdr} % This should be set AFTER setting up the page geometry
\pagestyle{fancy} % options: empty , plain , fancy
\renewcommand{\headrulewidth}{0pt} % customise the layout...
\lhead{}\chead{}\rhead{}
\lfoot{}\cfoot{\thepage}\rfoot{}

%%% SECTION TITLE APPEARANCE
\usepackage{sectsty}
\allsectionsfont{\sffamily\mdseries\upshape} % (See the fntguide.pdf for font help)
% (This matches ConTeXt defaults)

%%% ToC (table of contents) APPEARANCE
\usepackage[nottoc,notlof,notlot]{tocbibind} % Put the bibliography in the ToC
\usepackage[titles,subfigure]{tocloft} % Alter the style of the Table of Contents
\renewcommand{\cftsecfont}{\rmfamily\mdseries\upshape}
\renewcommand{\cftsecpagefont}{\rmfamily\mdseries\upshape} % No bold!

%%% END Article customizations

%%% The "real" document content comes below...

\title{EECS 233 HW8}
\author{Ben Pierce \\ \texttt{bgp12@case.edu}}
%\date{} % Activate to display a given date or no date (if empty),
         % otherwise the current date is printed 




%if anyone's reading this, any Latex tips/tricks would be greatly appreciated. 

\begin{document}
\maketitle
\title {GitHub: https://github.com/bp0017/CWRUEECS233/tree/master/HW8} 

\section{Question 1: Program Output}
\subsection{A}
\begin{lstlisting}
C:\Users\bp001\Documents\EECS223\HW8>java BinarySearcher
Searching for numbers in an array.
Is -1 in the array? No.
Is 0 in the array? No.
Is 1 in the array? No.
Is 2 in the array? Yes, at index [0].
Is 3 in the array? No.
Is 4 in the array? Yes, at index [1].
Is 5 in the array? No.
Is 6 in the array? Yes, at index [2].
Is 7 in the array? No.
Is 8 in the array? Yes, at index [3].
Is 9 in the array? No.
Is 10 in the array? Yes, at index [4].
Is 11 in the array? No.
Is 12 in the array? Yes, at index [5].
Is 13 in the array? No.
Is 14 in the array? Yes, at index [6].
Is 15 in the array? No.
Is 16 in the array? No.
Searching for 0 in an empty array:  Not found.
End of searching.
\end{lstlisting}

\subsection{B}
\begin{lstlisting}
Searching for numbers in an array.
Is -1 in the array?  Searching before 3  Searching before 1  Searching before 0 No.
Is 0 in the array?  Searching before 3  Searching before 1  Searching before 0 No.
Is 1 in the array?  Searching before 3  Searching before 1  Searching before 0 No.
Is 2 in the array?  Searching before 3  Searching before 1  Found at index 0
Is 3 in the array?  Searching before 3  Searching before 1  Searching after 0 No.
Is 4 in the array?  Searching before 3  Found at index 1
Is 5 in the array?  Searching before 3  Searching after 1 
Searching before 2 No.
Is 6 in the array?  Searching before 3  Searching after 1 
Found at index 2
Is 7 in the array?  Searching before 3  Searching after 1 
Searching after 2 No.
Is 8 in the array?  Found at index 3
Is 9 in the array?  Searching after 3 Searching before 5  
Searching before 4 No.
Is 10 in the array?  Searching after 3 Searching before 5  
Found at index 4
Is 11 in the array?  Searching after 3 Searching before 5  
Searching after 4 No.
Is 12 in the array?  Searching after 3 Found at index 5
Is 13 in the array?  Searching after 3 Searching after 5 No.
Is 14 in the array?  Searching after 3 Searching after 5 No.
Is 15 in the array?  Searching after 3 Searching after 5 No.
Is 16 in the array?  Searching after 3 Searching after 5 No.
Searching for 0 in an empty array:  Not found.
End of searching.
\end{lstlisting}
\section{Question 2}
\subsection{A}
\begin{lstlisting}
[1,3,2,4,6,5] Target = 1.
\end{lstlisting}
\subsection{B}
If the array was checked to be sorted before the search, the total runtime would be $N + \log_{2}N$ = $O(N)$ because the time to verify that an array is sorted is linear time, plus the time to search. This equals $O(N)$ because $N$ time is larger then $\log_{2}N$.
\section{Question 3}
\subsection{A}
\noindent\resizebox{\textwidth}{!}{
\begin{tabular}{ |c|c|c|c|c|c|c|c|c|c|c|c|c|c|c| } 
 \hline
 Key & Calculation & [0] & [1] & [2] & [3] & [4] & [5] & [6] & [7] & [8] & [9] & [10] & [11] & [12]\\ \hline
 1 & 1\%13 = 1&&1 &&&&&&&&&&&\\ \hline
 2 &2\%13 = 2&&1&2&&&&&&&&&&\\ \hline
 12&12\%13 = 12&&1&2&&&&&&&&&&12\\ \hline
 13&13\%13=0&13&1&2&&&&&&&&&&12\\ \hline
 14&14\%13=1&13&1&2&14&&&&&&&&&12\\ \hline
 130&130\%13=0&13&1&2&14&130&&&&&&&&12\\ \hline
 1212&1212\%13=3&13&1&2&14&130&1212&&&&&&&12\\ \hline
 1301&1301\%13=1&13&1&2&14&130&1212&1301&&&&&&12\\ \hline
 1300&1300\%13=0&13&1&2&14&130&1212&1301&1300&&&&&12\\ \hline
\end{tabular}
}

\subsection{B}
\noindent\resizebox{\textwidth}{!}{
\begin{tabular}{ |c|c|c|c|c|c|c|c|c|c|c|c|c|c|c| } 

 \hline
 Key & Calculation & [0] & [1] & [2] & [3] & [4] & [5] & [6] & [7] & [8] & [9] &  [10] & [11] & [12] \\ \hline
 1 & 1\%13 = 1 &&1&&&&&&&&&&&\\ \hline
 2 &2\%13 = 2&&1&2&&&&&&&&&&\\ \hline
 12&12\%13 = 12&&&&&&&&&&&&&12\\ \hline
 13&13\%13=0&13&1&2&&&&&&&&&&12\\ \hline
 14&14\%13=1, 14\%11=3&13&1&2&&14&&&&&&&&12\\ \hline
 130&130\%13=0, 130\%11=9&13&1&2&&14&&&&&130&&&12\\ \hline
 1212&1212\%13=3, 1212\%11=2&13&1&2&1212&14&&&&&130&&&12\\ \hline
 1301&1301\%13=1,1301\%11=3&13&1&2&1212&14&&&1301&&130&&&12\\ \hline
 1300&1300\%13=0, 1300\%11=2&13&1&2&1212&14&&1300&1301&&130&&&12\\ \hline

\end{tabular}
}
\section{Question 4}
\begin{lstlisting}
C:\Users\bp001\Documents\EECS223\HW8>java Table2
Trying index [1]
Trying index [2]
Trying index [12]
Trying index [0]
Trying index [1]
Trying index [2]
Trying index [3]
Trying index [0]
Trying index [1]
Trying index [2]
Trying index [3]
Trying index [4]
Trying index [12]
Trying index [0]
Trying index [1]
Trying index [2]
Trying index [3]
Trying index [4]
Trying index [5]
Trying index [1]
Trying index [2]
Trying index [3]
Trying index [4]
Trying index [5]
Trying index [6]
Trying index [0]
Trying index [1]
Trying index [2]
Trying index [3]
Trying index [4]
Trying index [5]
Trying index [6]
Trying index [7]
[0] [1] [2] [3] [4] [5] [6] [7] [8] [9] [10] [11] [12]
13 1 2 14 130 1312 1301 1300 null null null null 12
\end{lstlisting}
\section{Question 5}
\begin{lstlisting}
C:\Users\bp001\Documents\EECS223\HW8>java DoubleHash
Trying index [1] for 1
Trying index [2] for 2
Trying index [12] for 12
Trying index [0] for 13
Trying index [1] for 14
Trying index [4] for 14
Trying index [0] for 130
Trying index [9] for 130
Trying index [3] for 1212
Trying index [1] for 1301
Trying index [4] for 1301
Trying index [7] for 1301
Trying index [0] for 1300
Trying index [2] for 1300
Trying index [4] for 1300
Trying index [6] for 1300
[0][1][2][3][4][5][6][7][8][9][10][11][12]
13 1 2 1212 14 null 1300 1301 null 130 null null 12
\end{lstlisting}
\section{Question 6}
\subsection{A}
The run time for adding $N$ values to a hash table with no collisions is $O(N)$, because each value in $N$ is being inserted with a single assignment operation, so the total runtime is $1*N$ or $O(N)$
\subsection{B}
The runtime of finding a single value to a hash table with no collisions is $O(1)$, because it is a single operation. This is because the hash value is the index where the key is found, due to the assumption of no collisions.

\end{document}