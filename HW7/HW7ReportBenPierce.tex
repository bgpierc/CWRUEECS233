% !TEX TS-program = pdflatex
% !TEX encoding = UTF-8 Unicode

% This is a simple template for a LaTeX document using the "article" class.
% See "book", "report", "letter" for other types of document.
\documentclass[12pt]{article} % use larger type; default would be 10pt

%\documentclass{amsart}
\usepackage[utf8]{inputenc} % set input encoding (not needed with XeLaTeX)
\usepackage{hyperref}
\usepackage{listings}
\usepackage{movie15}
\usepackage{graphicx}

\hypersetup{
    colorlinks=true,
    linkcolor=blue,
    filecolor=magenta,      
    urlcolor=blue,
}

%%% Examples of Article customizations
% These packages are optional, depending whether you want the features they provide.
% See the LaTeX Companion or other references for full information.

%%% PAGE DIMENSIONS
\usepackage{geometry} % to change the page dimensions
\geometry{a4paper} % or letterpaper (US) or a5paper or....
% \geometry{margin=2in} % for example, change the margins to 2 inches all round
% \geometry{landscape} % set up the page for landscape
%   read geometry.pdf for detailed page layout information

\usepackage{graphicx} % support the \includegraphics command and options

% \usepackage[parfill]{parskip} % Activate to begin paragraphs with an empty line rather than an indent

%%% PACKAGES
\usepackage{booktabs} % for much better looking tables
\usepackage{array} % for better arrays (eg matrices) in maths
\usepackage{paralist} % very flexible & customisable lists (eg. enumerate/itemize, etc.)
\usepackage{verbatim} % adds environment for commenting out blocks of text & for better verbatim
\usepackage{subfig} % make it possible to include more than one captioned figure/table in a single float
% These packages are all incorporated in the memoir class to one degree or another...

%%% HEADERS & FOOTERS
\usepackage{fancyhdr} % This should be set AFTER setting up the page geometry
\pagestyle{fancy} % options: empty , plain , fancy
\renewcommand{\headrulewidth}{0pt} % customise the layout...
\lhead{}\chead{}\rhead{}
\lfoot{}\cfoot{\thepage}\rfoot{}

%%% SECTION TITLE APPEARANCE
\usepackage{sectsty}
\allsectionsfont{\sffamily\mdseries\upshape} % (See the fntguide.pdf for font help)
% (This matches ConTeXt defaults)

%%% ToC (table of contents) APPEARANCE
\usepackage[nottoc,notlof,notlot]{tocbibind} % Put the bibliography in the ToC
\usepackage[titles,subfigure]{tocloft} % Alter the style of the Table of Contents
\renewcommand{\cftsecfont}{\rmfamily\mdseries\upshape}
\renewcommand{\cftsecpagefont}{\rmfamily\mdseries\upshape} % No bold!

%%% END Article customizations

%%% The "real" document content comes below...

\title{EECS 233 HW7}
\author{Ben Pierce \\ \texttt{bgp12@case.edu}}
%\date{} % Activate to display a given date or no date (if empty),
         % otherwise the current date is printed 




%if anyone's reading this, any Latex tips/tricks would be greatly appreciated. 

\begin{document}
\maketitle
\title {GitHub: https://github.com/bp0017/CWRUEECS233/tree/master/HW7} 

\section{Question 1}

a)\\
\includegraphics[scale = 0.5]{a.png}

b)\\
\includegraphics[scale = 0.5]{b.png}

c)\\
\includegraphics[scale =0.5]{c.png}

d)\\
\includegraphics[scale = 0.5]{d.png}

e)\\
\includegraphics[scale =0.5]{e.png}

f)\\
\includegraphics[scale = 0.5]{f.png}

g)\\
\includegraphics[scale = 0.5]{g.png}

h)\\
\includegraphics[scale = 0.5]{h.png}

i)\\
\includegraphics[scale = 0.5]{i.png}

j)\\
\includegraphics[scale = 0.5]{j.png}

k)\\
\includegraphics[scale = 0.5]{k.png}

l)\\
\includegraphics[scale = 0.5]{l.png}

m)\\
\includegraphics[scale = 0.5]{m.png}

n)\\
\includegraphics[scale = 0.5]{n.png}

\section{Question 2}
\begin{lstlisting}
C:\Users\bp001\Documents\EECS223\HW7>java Heap
Adding values...
[0] 8
[1] 5
[2] 7
[3] 4
[4] 3
[5] 2
[6] 6
[7] 1
Removing... 8
[0] 7
[1] 5
[2] 2
[3] 4
[4] 3
[5] 1
[6] 6
Removing... 7
[0] 6
[1] 5
[2] 2
[3] 4
[4] 3
[5] 1
Removing... 6
[0] 5
[1] 4
[2] 2
[3] 1
[4] 3
Removing... 5
[0] 4
[1] 3
[2] 2
[3] 1
Removing... 4
[0] 3
[1] 1
[2] 2
Removing... 3
[0] 2
[1] 1
\end{lstlisting}

\section{Question 3}
\includegraphics{graph.png}
\\blue= IntArrayBag, red = Heap
\\y-axis in ms
\section{Question 4}
a) The add operation for the IntArrayBag class is $O(1)$ because it adds a single element to the end of an array with one assignment.
\\b) The remove operation for the IntArrayBag where the element is the last item of the array is $O(N)$ because the remove method has to search through the entire array of length $N$.
\\c) The sum of the add and remove operations for the IntArrayBag class is $O(N)$, because as $N$ gets larger, the $O(1)$ time of the add operation becomes less important.
\\d) The add method for the Heap class where the element added is larger then the root is $O(log N)$ because the element needs to shift the entire height of the tree, which is $log N$ where N is the height of the tree.
\\e) The removeMax() method for the Heap class is worst-case $O(log N)$ because at the worst case, the element swapped to the root of the tree has to sift downwards through the whole tree of height $log N$.
\\f) The sum of the add and removeMax methods for the Heap class is $O(log N)$ because two methods with time complexity $log N$ are summed together, causing the resulting time complexity to be $2*log N$, which simplifies to $O(log N)$.

\section{Question 5}
The experimental results do not really agree with the expected results; for large N the heap times would be expected to plateau, whereas the IntArrayBag times would be expected to increase in a linear fashion. The experimental results have both at constant time. This is probably due to experimental (programming) error.





\end{document}